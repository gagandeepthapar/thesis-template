\chapter{The Literature}
%\Chapter{Literature Review}{A Narrative}

\section{A Narrative and Motivation}
\par \qquad Observing and measuring nature are the basic tools we can use to characterize the world around us.
We can use increasingly accurate sensors to measure different properties of the universe, however, we must come to terms with a fundamental reality.
True values of these measurements are non-observable, that is, we can never verify our measurements in nature and are doomed to contain some error in our work.
We can attribute these errors to our measurement techniques and the idea that we only have discrete tools trying to measure a continuous reality.

\par \qquad In space, a key property to measure is the attitude, or orientation, of our spacecraft.
The direction we point in is imperative for a variety of reasons such as pointing for power generation, pointing to observe interesting phenomena, and pointing to survive the environment.
One such sensor to measure our attitude is the star tracker; a clever technology that utilizes the positions of the stars in the universe to ascertain its attitude.
The star tracker, similar to any other sensor, is ladened with pitfalls in its measurement process and, due to a variety of factors, is prone to error.
It's important to analyze the error propagation within the star tracker as, given a satisfactory model, a star tracker can be optimized for multiple considerations such as cost, speed, and accuracy.

\par \qquad The literature offers several suggestions in analyzing the error propagation within the sensor such as hardware considerations, "thermal drift, optical aberration, detector noise"\cite{systematic_error_analysis_of_star_tracker_centroiding}, and systematic algorithmic errors. However, before we can analyze where error is prone to occur, we must first understand what a star tracker is and how they work.

\section{What is a Star Tracker}
\par \qquad A star tracker is a sensor installed on spacecraft which utilizes external references, being the stars in the sky, to determine its attitude.