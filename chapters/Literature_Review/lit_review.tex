\chapter{The Literature}
%\Chapter{Literature Review}{A Narrative}

\section{A Narrative and Motivation}
\par \qquad Observing and measuring nature are the basic tools we can use to characterize the world around us.
We can use increasingly accurate sensors to measure different properties of the universe, however, we must come to terms with a fundamental reality.
True values of these measurements are non-observable, that is, we can never verify our measurements in nature and are doomed to contain some error in our work.
We can attribute these errors to our measurement techniques and the idea that we only have discrete tools trying to measure a continuous reality.

\par \qquad In space, a key property to measure is the attitude, or orientation, of our spacecraft.
The direction we point in is imperative for a variety of reasons such as pointing for power generation, pointing to observe interesting phenomena, and pointing to survive the environment.
One such sensor to measure our attitude is the star tracker; a clever technology that utilizes the positions of the stars in the universe to ascertain its attitude.
The star tracker, similar to any other sensor, is ladened with pitfalls in its measurement process and, due to a variety of factors, is prone to error.
It's important to analyze the error propagation within the star tracker as, given a satisfactory model, a star tracker can be optimized for multiple considerations such as cost, speed, and accuracy.

\par \qquad The literature offers several suggestions in analyzing the error propagation within the sensor such as hardware considerations, "thermal drift, optical aberration, detector noise"\cite{systematic_error_analysis_of_star_tracker_centroiding}, and systematic algorithmic errors. However, before we can analyze where error is prone to occur, we must first understand what a star tracker is and how they work.

\section{What is a Star Tracker}
\par \qquad A star tracker is an optical sensor installed on spacecraft which utilizes external references, being the stars in the sky, to determine its attitude.
The star tracker has 3 fundamental steps in attitude determination: image acquisition, centroiding, and identification; each with its own set of considerations and locations for error to exist.
In general, the star tracker takes a picture of the stars during the image acquisition phase, determines where in the image all of the stars are during the centroiding phase, and identifies and compares the stars in the image to a pre-generated catalog during the identification phase. Here, we will discuss what happens in each phase and where possible errors may arise.

\subsection{Image Acquisition}
\par \qquad During this phase, the star tracker captures an image of the stars in the sky.
It's common in the literature to model the star tracker as a pinhole camera to simplify the system into basic vector expressions which can easily be manipulated.
From here, we can use intrinsic properties of the camera, such as the focal length and focal plane size, to determine where a star is and some key properties such as inter-star angles.

\par \qquad The Image Acquisition phase, however, is where the hardware considerations of the star tracker are important.
Components such as the lens shape, focal length, and focal plane are all components of interest and directly influence this phase.
Errors during this phase are typically due to environmental effects such as the thermal distortion of the lens or the radiation environment corrupting the focal sensor. 

\subsection{Centroiding}
\par \qquad Centroid

\subsection{Identification}
\par \qquad ID algos

\par \qquad Separate from the 3 main operation phases, however, are the errors that can exist as part of the intrinsic properties of the camera.
Considerations in the accuracy of the focal length, "flatness" of the focal plane, and relative offset of the focal plane and boresight all contribute to the accuracy limits of a given star tracker.
These errors are often reduced or eliminated through a calibration procedure pre-launch.