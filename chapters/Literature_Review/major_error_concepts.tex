\chapter{Major Concepts in Star Tracker Error Analysis}
\begin{itemize}
    \item Different steps in the Star Tracker process
    \begin{itemize}
        \item Centroiding
        \item Identification
        \item Attitude Determination
    \end{itemize}
    \item Errors can exist on several levels
    \begin{itemize}
        \item Hardware (permanent non-varying)
        \item Hardware noise (varies, function of hardware)
        \begin{itemize}
            \item CCD noise 
            \item Dark Current 
        \end{itemize}
        \item Environment (varies, can affect hardware which)
        \begin{itemize}
            \item Thermal Environment on Lens
            \item Radiation Environment on focal plane 
            \item Atomic Oxygen on lens 
        \end{itemize}
        \item Dynamics (varies, can affect centroiding process)
        \item Different identification algorithms can yield different results
        \item Wahba's problem which will affect attitude determination from identification algorithm
    \end{itemize}
    \item "The most important factors that affect star tracker accuracy include \textbf{thermal drift, optical aberration, detector noise, and systematic error of star image centroid estimation algorithm}"\cite{systematic_error_analysis_of_star_tracker_centroiding}
    \item Most of the literature describes the centroiding/calibration being the prime mover of error; on-orbit effects are rarely discussed although do pose additional error in attitude determination
    \item 
\end{itemize}