\section{Background Knowledge and Context}

\subsection*{CubeSats and Pointing}

\par \qquad Since their inception in 1999, CubeSats have been employed for a variety of mission types. 
From measuring elements in the Earth's exosphere, to testing vibration and dampening experiments in microgravity, to experimenting accelerated deorbit times via dragsail \cite{PolySat}, it's clear to see that CubeSats are not pigeonholed into a single mission archetype.
However, as the technology has matured and missions become more complex, their requirements, especially for pointing, have become stricter.

\par \qquad CubeSats, being developed at a university, continue to be designed at universities such as California Polytechnic State University - San Luis Obispo (Cal Poly SLO) or University of California - Davis (UC Davis).
The initial design made it possible for organizations with limited resources and funding to still be able to develop functional spacecraft with unique mission objectives. 
While CubeSats have emerged into the greater market and can be developed with upwards of \$1 Million, typical budget-constrained or university CubeSat programs can be expected to spend on the order of \$50,000 - \$200,000 \cite{SmallSatMarket}.
While a large sum of funds, it should be noticed that a CubeSat is a complicated amalgamation of instruments, supporting avionics, structure, and software, all of which is still expected to undergo and survive testing such as vibration and thermal vacuum tests to comply with launch vehicle standards.
Each of these subsystems and requirements requires some amount of the total funding.
As a consequence, funds allocated towards supporting sensors, such as attitude determination sensors, are typically a small portion of the total funding. 
Despite the limited funding available for such CubeSat programs, technology demands and mission requirements have increased in complexity, creating a need for more affordable component alternatives.
Pointing requirements, in specific, have become increasingly strict.
As CubeSats continue to demonstrate their capabilities for missions, earth-sensing missions and inter-satellite communication systems have slowly been outsourced from traditional, larger spacecraft to the modern CubeSat.
As a result, pointing has become increasingly important to consider and control.

\subsection*{QuakeSat and CLICK}
\par \qquad QuakeSat was a mission which would measure and record earthquakes on the surface \cite{QuakeSat}.
The team from Space Systems Development Laboratory (SSDL) reported, for mission success, they used a wide-angle light sensor and magnetometer; they "were not ideal for attitude determination"\cite{QuakeSat}.
Stories like QuakeSat characterize the the early missions of CubeSats - where attitude was not a major concern and, as a consequence, became afterthoughts in the design cycle.
With more modern missions, however, attitude has become the forcing function for mission success.
The CubeSat Laser Infrared CrosslinK (CLICK) was a technology demonstration for an inter-satellite laser-based communication system from MIT\cite{CubeSatCLICK}.
The demonstration, through analysis of the inter-satellite distance and relative motion, required the pair of CubeSats to point in a single axis to within $\pm$ 5.18 arcsec of each other.
Another optical mission, ASTERIA, looks to optimize beam pointing control and requires 3.3 arcsec \cite{OnOrbitBeamCalibration} of pointing accuracy to complete its objective.
The trend of CubeSats taking on complex technology demonstration and requiring strict pointing requirements has only and will only increase as the confidence in CubeSat technology improves.
A worrying observation, however, is that the list of sensors able to meet requirements is starting to narrow down. 
Without a suitable sensor, CubeSat capabilities can stagnate indefinitely.

\par \qquad With other sensors, sun sensors, for example, accuracy begins to fall.
Sun sensors can be as accurate as 0.1 deg \cite{SunSensorA}, or 360 arcsec, and cost as much as \$9,000 \cite{CubeSatShop_SunSensor}. 
While sun sensors were once the standard for attitude determination on budget-constrained CubeSats, it's clear that they can no longer fulfill the requirements being imposed on modern day CubeSat mission.
On the opposite end of the spectrum, however, star trackers can be as accurate as 5 arcsec but as expensive as \$120,000 \cite{RocketLabStarTracker} or more.
CubeSats - particularly developed in university settings or those that are budget constrained - are typically underfunded to afford such star trackers.
Other attitude determination alternatives include magnetometers (whose accuracy can vary depending on the orbit type and can be as accurate as 0.01 degrees \cite{Magnetometer}), horizon sensors (which can be accurate to approximately 0.1 degrees \cite{HorizonSensor}), or passive systems such as gravity-gradient which are only accurate to approximately 15 degrees \cite{GravityGradient} depending on the construction and mass properties of the spacecraft.
While alternatives may be cheaper, it's clear to see that the performance gap compared to star trackers is great and needs to be closed. 
This gap can be filled by trading on star tracker performance for cost, allowing for more affordable alternative options while still outperforming less expensive sensors.
Parameterizing the error in star trackers based on selected hardware and expected environmental conditions are one such method to determine how to perform the performance trade.

\begin{center}
    \emph{Figure plotting star tracker and other attitude sensor cost vs performance}
\end{center}

\subsection*{Parameterization of Error}

\par \qquad While other approaches in determining how to develop more affordable star trackers are viable, analyzing the error propagation allows us to determine additional information about the system and develop a tool for CubeSat developers.
By looking at each physical step of the star tracker from end-to-end, a complete mathematical model of the star tracker can be deduced, including, but not limited to, where errors occur for one reason or another.
For example, by identifying flaws in the camera construction such as the normality of the boresight vector or the tilt of the focal array, immediate errors can be characterized and modeled as the ideal model for centroiding would be based on a skewed representation of the celestial sphere.
By incorporating different effects, specifically the camera hardware and noise, the algorithmic errors, and the leading environmental errors in low earth orbit (LEO), a model can be devised where a given set of inputs (i.e., from a list of mission requirements) can be used to devise how accurate a theoretical star tracker would be.
This black box would enable CubeSat developers to determine if investing in a low-cost star tracker will be sufficient for the mission without needing to experimentally test for it using their expected mission parameters.

\par \qquad Investigating the error propagation also enables us to determine where the most errors are prone to occur and where it is worth the time and resources to optimize a process. 
By fully understanding and parameterizing the star tracker process, a greater understanding of how to model the star tracker accuracy in on-orbit conditions can be determined and used to drive decision making for attitude determination sensor selection without \emph{a priori} experimental information.

\par \qquad Alternative methods in reducing the cost of the star tracker include optimizing hardware or writing custom firmware for the optical module.
While worthwhile efforts, the same pitfalls of high development costs still persist.
Optimizing the hardware will still require some knowledge of how the optical system works and how different decisions and hardware permutations will affect the estimated accuracy.
This black box devised through analyzing error propagation allows developers to take the guess work out of the hardware selection and reduces the development cost and time in broad iteration.
Software optimization is another alternative, however, without a great understanding of the physics, image analysis process, and flight software knowledge, developing personalized algorithms will prove to be burdensome and can still lead to expensive development cycles.
Analyzing the error propagation through a theoretical lens allows us to specifically determine where errors, or losses, are generated and how hardware selection can help mitigate errors. 
It also enables us to selectively allow errors to persist by choosing affordable hardware, thereby reducing the total cost of the star tracker.
It should be clear by now that, due to the rapid maturation of CubeSat technology and capability, a demand for equally capable attitude determination sensors needs to be met, especially one that fits within the confines of a typical CubeSat budget.

\par \qquad The literature offers several suggestions in analyzing the error propagation within the sensor such as hardware considerations, "thermal drift, optical aberration, detector noise"\cite{systematic_error_analysis_of_star_tracker_centroiding}, and systematic algorithmic errors.
However, before we can analyze where error is prone to occur, we must first understand what a star tracker is and how they work.