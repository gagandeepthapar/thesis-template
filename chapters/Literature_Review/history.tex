\section{CubeSats and Pointing}

\par \qquad Since their inception in 1999, CubeSats have been employed for a variety of mission types. 
From measuring elements in the Earth's exosphere to testing vibration and dampening experiments in microgravity, to experimenting accelerated deorbit times via dragsail \cite{PolySat}, it's clear to see that CubeSats are not pigeonholed into a single mission archetype.
However, as the technology has matured and missions become more complex, their requirements, especially for pointing, have become stricter.

\par \qquad CubeSats, being developed at a university, continue to be designed at universities such as California Polytechnic State University - San Luis Obispo (Cal Poly SLO) or University of California - Davis (UC Davis).
The initial design made it possible for organizations with limited resources and funding to still be able to develop functional spacecraft with unique mission objectives. 
While CubeSats have emerged into the greater market and can be developed with upwards of \$1 Million, typical budget-constrained or university CubeSat programs can be expected to spend on the order of \$50,000 - \$200,000 \cite{SmallSatMarket}.
While a large sum of funds, it should be noticed that a CubeSat is a complicated amalgamation of instruments, supporting avionics, structure, and software, all of which is still expected to undergo and survive testing such as Vibration and Thermal Vacuum tests to comply with launch vehicle standards, each costing a significant amount.
As a consequence, funds allocated towards supporting sensors, such as attitude determination sensors, are typically a small portion of the total funding. 
Despite the limited funding available for such CubeSat programs, technology demands and mission requirements have increased in complexity, creating a need for more affordable component alternatives.
Pointing requirements, in specific, have become increasingly strict.
As CubeSats continue to demonstrate their capabilities for missions, earth-sensing missions and inter-satellite communications have slowly been outsourced from traditional, larger spacecraft to the modern CubeSat.
As a result, pointing has become increasingly important to consider and control.

\par \qquad QuakeSat was a mission which, through the use of magnetometers, would measure and record earthquakes on the surface \cite{QuakeSat}.
The team from Space Systems Development Laboratory (SSDL) reported, for mission success, they used a wide-angle light sensor; they "were not ideal for attitude determination"\cite{QuakeSat}.
Stories like QuakeSat characterize the the early missions of CubeSats - where attitude was not a major concern and, as a consequence, became afterthoughts in the design cycle.
With more modern missions, however, attitude has become the forcing function for mission success.
The CubeSat Laser Infrared CrosslinK (CLICK) was a technology demonstration for an inter-satellite laser-based communication system from MIT\cite{CubeSatCLICK}.
The demonstration, through analysis of the inter-satellite distance and relative motion, required the pair of CubeSats to point in a single axis to within $\pm$ 5.18 arcsec of each other.
Another optical mission, ASTERIA, looks to optimize beam pointing control and requires 3.3 arcsec \cite{OnOrbitBeamCalibration} of pointing accuracy to complete its objective.
The trend of CubeSats taking on complex technology demonstration and requiring strict pointing requirements has only and will only increase as the confidence in CubeSat technology improves.
A worrying observation, however, is that the list of sensors able to meet requirements is starting to narrow down. 
Without a suitable sensor, CubeSat capabilities can stagnate indefinitely.

\par \qquad With other sensors, such as sun sensors and magnetometers, accuracy begins to fall.
Sun sensors can be as accurate as 0.1 deg \cite{SunSensorA}, or 360 arcsec, and cost as much as \$9,000 \cite{CubeSatShop_SunSensor}. 
While sun sensors were once the standard for attitude determination on budget-constrained CubeSats, it's clear that they can no longer fulfill the requirements being imposed on modern day CubeSat mission.
On the opposite end of the spectrum, however, star trackers can be as accurate as 5 arcsec but as expensive as \$120,000 \cite{RocketLabStarTracker}.
CubeSats - particularly developed in university settings or those that are budget constrained - are typically underfunded to afford such star trackers. 
It's clear to see that the gap between star trackers and other attitude determination sensors needs to be closed. 
This gap can be filled by trading on star tracker performance for cost, allowing for more affordable alternative options while still outperforming less expensive sensors.
Parameterizing the error in star trackers based on selected hardware and expected environmental conditions are one such method to determine how to perform the performance trade.

\section{Parameterization of Error}

\par \qquad test