\par \qquad Observing and measuring nature are the basic tools we can use to characterize the world around us.
We can use increasingly accurate sensors to measure different properties of the universe, however, we must come to terms with a fundamental reality.
True values of these measurements are non-observable, that is, we can never verify our measurements in nature and are doomed to contain some error in our work.
We can attribute these errors to our measurement techniques and the idea that we only have discrete tools trying to measure a continuous reality.

\par \qquad In space, a key property to measure is the attitude, or orientation, of our spacecraft.
The direction we point in is imperative for a variety of reasons such as pointing for power generation, pointing to observe interesting phenomena, and pointing to survive the environment.
One such sensor to measure our attitude is the star tracker; a clever technology that utilizes the positions of the stars in the universe to ascertain its attitude.
The star tracker, similar to any other sensor, is ladened with pitfalls in its measurement process and, due to a variety of factors, is prone to error.
Before we move forward, we must determine why analyzing the error propagation in a star tracker is worth doing; what will benefit from studying how star trackers err.

\section*{CubeSats}
\par \qquad CubeSats, developed by the Cal Poly CubeSat Lab in 1999 by Prof. Jordi Puig-Suari and Prof. Bob Twiggs, have long been used as a bus for university programs and commercial businesses to fly experiments in Low Earth Orbit, or, LEO.
CubeSats typically use COTS, or commercial-off-the-shelf, components as they're characterized by their ease and low-cost of development.
While CubeSats have typically been used for simpler observation missions, the technology has been maturing and, as a consequence, as have their mission requirements.
Attitude determination and control have been an increasingly popular requirement imposed on CubeSats as a means to accomplish more complex missions i.e., optical-communication demonstrations (SOTA, OSIRISv2)\cite{OpticalCommsInSpace}.
With the attitude determination requirements becoming increasingly stringent, the sensor selection space converges to only a few select types considering the constraints CubeSat developers are forced to deal with such as lack of payload volume, limited power options, and especially total cost.
Star trackers are a prime example of a sensor that fits the mission criteria but are overlooked due to their cost.
Star trackers for CubeSats can start at \$30,000 and only go up from there; an cost that can sometimes equal if not be greater than the total budget for a mission.
However, with attitude determination requirements becoming stricter, a need for a low-cost solution emerges.

\par \qquad The fundamental question to ask is whether or not a star tracker, traded on performance for cost, is a viable solution for CubeSat attitude determination in the future.
This thesis aims to answer the question by analyzing where errors exist, how they propagate, and how different hardware can influence the expected attitude accuracy.


