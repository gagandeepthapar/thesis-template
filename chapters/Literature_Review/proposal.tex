\par \qquad While not expected to be included in the final thesis (at least in this form), the proposal is left here for completeness.

\section{Motivation}
\begin{itemize}
    \item CPCL already developing a star tracker to deploy in orbit and open sourcing the software 
    \item Would be useful for other universities and low-budget developers to be able to determine if star tracker is worth putting money into 
    \item Allows cubesats to stay competitive in the maturing spacecraft market 
\end{itemize}

\section{Contribution to the Field}
\begin{itemize}
    \item No other tool to determine star tracker accuracy 
    \item Focused on CubeSats i.e., low-cost, low-size 
    \item Allows cubesat developers to estimate their accuracy to determine if a low cost star tracker is inline with the pointing direction
\end{itemize}

\section{Scope and Statement of Work}
\begin{itemize}
    \item Determine major effects to consider
    \begin{itemize}
        \item Hardware
        \begin{itemize}
            \item Normality of boresight
            \item Focal plane distortions
            \item Size of focal array
            \item CCD Noise (ADC Noise, Dark Current)
        \end{itemize}
        
        \item Environment
        \begin{itemize}
            \item Thermal Environment (Distortions, Cycling)
            \item Body Rates
        \end{itemize}

        \item Algorithmic
        \begin{itemize}
            \item Centroiding Errors 
            \item Identification Errors 
            \item QUEST Errors 
        \end{itemize}
    \end{itemize}
    \item Discover or determine mathematical models or relations between effects and star tracker accuracy on each step in star tracker process (image capture, centroiding, etc)
    \item Create function/black box for inputs to feed through and spit out star tracker accuracy 
    \item Use CPCL star tracker hardware + software as case study
    \item Determining variables to randomize for Monte Carlo Analysis 
    \begin{itemize}
        \item Absortivity/Emissivity for thermals 
        \item CCD Noise 
        \item Focal Plane distortions 
        \item Boresight Normality
        \item Radiation
        \item Centroiding Accuracy (likely discrete set)
        \item Identification Accuracy (likely discrete set)
    \end{itemize}
    \item Determining variables to keep consistent
    \begin{itemize}
        \item QUEST performance 
        \item LEO altitude ~ 90min orbit period 
        \item CubeSat sizes (1-3U)
        \item  
    \end{itemize}
\end{itemize}

\section{Schedule}
\begin{itemize}
    \item Through Fall Quarter:
    \begin{itemize}
        \item Finish Centroiding Algorithm 
        \item Analyze Centroiding Performance
        \item Finalize list of effects to consider
        \item Find environmental models to use for black box
        \item Propose
    \end{itemize}
    \item Through Winter Quarter:
    \begin{itemize}
        \item Find additional models to complete larger order error sources 
        \item Combine models to create "black box" of inputs that outputs estimated accuracy
        \item Requires validation
        \item Analyze identification performance
        \item Devise and validate cost functions 
        \item Determine optimal solutions for the cost functions  
    \end{itemize}
    \item Through Spring Quarter
    \begin{itemize}
        \item Finish analysis of all models
        \item Write up thesis
        \item Defend
    \end{itemize}
\end{itemize}

\section{Methodology}
\begin{itemize}
    \item Find models relating effects to star tracker accuracy 
    \item Combine models where appropriate (i.e., thermal effect on focal length where $f$ becomes a function instead of a constant)
    \item Determine how effects affect the star tracker (i.e., during the centroiding process will propagate from there on)
    \item Determine how accurate each physical process can be at its max as a function of the different effects
    \item Create black box based on Monte Carlo analysis looking at discrete sets of camera characteristics and continuous sets of physical properties to estimate star tracker accuracy
    \item Devise cost function based on community needs and use black box to suggest optimal camera and environment opportunities 
\end{itemize}

\section{Success Criteria}
\begin{itemize}
    \item Black box where various inputs (camera properties, environment, etc.) are fed to estimate star tracker accuracy with some confidence level
    \item Black box is opened up to show models and how it affects each step in attitude determination 
\end{itemize}

\section{Expected Challenges}
\begin{itemize}
    \item Finding models
    \item Combining models 
    \item Devising the cost function 
\end{itemize}

\par \qquad The literature offers several suggestions in analyzing the error propagation within the sensor such as hardware considerations, "thermal drift, optical aberration, detector noise"\cite{systematic_error_analysis_of_star_tracker_centroiding}, and systematic algorithmic errors.
However, before we can analyze where error is prone to occur, we must first understand what a star tracker is and how they work.